
\clearpage
\phantomsection

\addcontentsline{toc}{subsection}{SETX}
\label{insn:setx}
\subsection*{SETX: Retrieve an integer from the integer table.}

\subsubsection*{Format}
\texttt{SETX rd index}

\begin{center}
  \begin{bytefield}[endianness=big,bitformatting=\scriptsize]{32}
    \bitheader{0,7,8,15,16,31}
    \bitbox{8}{0x25}
    \bitbox{8}{rd}
    \bitbox{16}{index}
  \end{bytefield}
\end{center}
\subsubsection*{Description}

Look up an integer value stored in the DIF integer table and place it
into rd. This instruction performs no bounds checking.


\subsubsection*{Pseudocode}

\begin{lstlisting}[language=Python]
def SETX(state, rd, index):
    state.registers[rd] = state.tables.integers[index]
\end{lstlisting}
